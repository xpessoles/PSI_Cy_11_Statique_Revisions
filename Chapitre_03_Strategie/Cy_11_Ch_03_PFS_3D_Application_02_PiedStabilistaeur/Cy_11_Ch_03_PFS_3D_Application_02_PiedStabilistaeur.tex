% "{'classe':('PSI'),'chapitre':'stat_pfs_3d','type':('application'),'titre':'Pied stabilisateur', 'source':'Equipe PT La Martin','comp':('B2-14','C1-05','C2-07'),'corrige':True}"
%\setchapterimage{fig_00.jpg}
\chapter*{Application \arabic{cptApplication} \\ 
Pied stabilisateur -- \ifprof Corrigé \else Sujet \fi}
\addcontentsline{toc}{section}{Application \arabic{cptApplication} : Pied stabilisateur-- \ifprof Corrigé \else Sujet \fi}

\iflivret \stepcounter{cptApplication} \else
\ifprof  \stepcounter{cptApplication} \else \fi
\fi

\setcounter{question}{0}
\marginnote{Equipe La Martinière Monplaisir.}
\begin{marginfigure}
\includegraphics[width=\linewidth]{fig_01}
\end{marginfigure}

%\subsection*{Pied stabilisateur}
\ifprof
\else
On s’intéresse à un pied stabilisateur d’un engin de chantier.
La figure 1 représente l’un des 4 pieds stabilisateurs d’un engin de chantier. Chaque pied est composé d’un patin (5), de deux barres (3) et (4) et d’un vérin hydraulique (1+2) (1=corps, 2=piston). Les barres sont articulées en A et B sur le bâti (0) de l’engin et en D sur le patin. Toutes les liaisons sont considérées comme des liaisons rotules et la liaison en D est commune aux pièces (2), (3), (4) et (5).

\begin{center}
\includegraphics[width=.6\linewidth]{fig_02}\hfill
\end{center}

Extrait du diagramme des exigences :
\begin{center}
\begin{tabular}{llll}
\hline
Exigences & Critères & Niveau & Limite \\
\hline
Adaptation au vérin hydraulique	& Effort transmissible par le vérin  & \SI{60}{kN} & Maxi \\
Dimensionnement RdM & Action dans les barres & \SI{25}{kN} & Maxi \\
\hline
\end{tabular}
\end{center}

On donne $\vectf{\text{ext}}{5} = F\vect{z}$ avec $F = \SI{30000}{N}$.

La dimension $a=\SI{400}{mm}$, les points $A$, $B$ et $C$ sont dans le plan $\left(O,\vect{y},\vect{z}\right)$.
On a : $\vect{OD} = 6a\vect{x}$, $\vect{OC} = 5a\vect{z}$, $\vect{OB} = 2a\vect{y}+a\vect{z}$, $\vect{OA} = - 2a\vect{y}+a\vect{z}$. 

On pourra noter : $\vect{CD} = L_{12}\vect{x_{12}}$, $\vect{AD} = L_3\vect{x_3}$, $\vect{BD} = L_4\vect{x_4}$.
\fi

\question{Analyser le mécanisme (calculer le degré d'hyperstisme) et proposer les étapes de résolution du problème de détermination des efforts dans les liaisons.}
\ifprof
\begin{corrige}
Les pièces \{1+2\}, 3 et 4 sont bi-rotulées. Donc il existe une mobilité de rotation propre (respectivement autour de $\vect{x_{12}}$, $\vx{3}$, $\vx{4}$.
La pièce 5 peut faire 3 mouvements de rotation.

Au final, $m=6$.

Il y a 6 liaisons rotules; donc 18 inconnues statiques. 

Il y a 4 pièces que l'on peut isoler, donc 24 équations statiques. 

Au final, $h = m-E_S+I_S = 6 -24 + 18 = 0$. Le système est isostatique. On peut calculer toutes les actions mécaniques.

\vspace{.5cm}

Le torseur d'une liaison rotule, en statique est un glisseur. 

Les ensembles \{1+2\}, 3 et 4 sont tous 3 soumis à deux glisseurs. 

En appliquant le PFS à chacun de ces ensembles, on a : 
$\torseurstat{T}{2}{5} = \torseurl{F_2\vect{x_{12}}}{\vect{0}}{D}$,
$\torseurstat{T}{3}{5} = \torseurl{F_3\vect{x_3}}{\vect{0}}{D}$,
$\torseurstat{T}{4}{5} = \torseurl{F_4\vect{x_4}}{\vect{0}}{D}$.

On isole alors 5 soumis à 4 actions mécaniques. 
On applique le TRS en projection sur $\vx{0}$, $\vy{0}$, $\vz{0}$. 
\end{corrige}
\else
\fi


\question{Calculer les actions exercées par le patin sur les barres et le vérin.}
\ifprof
\begin{corrige}
On a $\vect{x_{12}} = \dfrac{\vect{CD}}{||\vect{CD}||}$
$ =\dfrac{1}{\sqrt{36a^2+25a^2}} \begin{pmatrix}6a \\ 0 \\ -5a \end{pmatrix}$ 
$= \dfrac{1}{\sqrt{61}} \begin{pmatrix}6 \\ 0 \\ -5 \end{pmatrix}$.


On a $\vect{x_{3}} = \dfrac{\vect{AD}}{||\vect{AD}||}$
$ =\dfrac{1}{\sqrt{4a^2+a^2 +36a^2}} \begin{pmatrix}6a \\ 2a \\ -a \end{pmatrix}$ 
$= \dfrac{1}{\sqrt{41}} \begin{pmatrix}6 \\ 2 \\ -1 \end{pmatrix}$.

On a $\vect{x_{4}} = \dfrac{\vect{BD}}{||\vect{BD}||}$
$ =\dfrac{1}{\sqrt{36a^2+4a^2 +a^2}} \begin{pmatrix}6a \\ -2a \\ -a \end{pmatrix}$ 
$= \dfrac{1}{\sqrt{41}} \begin{pmatrix}6 \\ -2 \\ -1 \end{pmatrix}$.

En isolant 5 et en appliquant le TRS, on a donc :
$
\left\{
\begin{array}{l}
\dfrac{6F_2}{\sqrt{61}}	+\dfrac{6F_3}{\sqrt{41}}	+\dfrac{6F_4}{\sqrt{41}}=0 \\
0		+\dfrac{2F_3}{\sqrt{41}}	-\dfrac{2F_4}{\sqrt{41}}=0 \\
-\dfrac{5F_2}{\sqrt{61}}	-\dfrac{F_3}{\sqrt{41}}		-\dfrac{F_4}{\sqrt{41}} + F=0 \\
\end{array}
\right.
$

D'après la résultante sur $\vect{y}$, $F_3=F_4$; donc 
$
\left\{
\begin{array}{l}
\dfrac{6F_2}{\sqrt{61}}	+\dfrac{12F_3}{\sqrt{41}} =0 \\
-\dfrac{5F_2}{\sqrt{61}}	-\dfrac{2F_3}{\sqrt{41}} + F=0 \\
\end{array}
\right.
$

Par suite, 
$
\left\{
\begin{array}{l}
\dfrac{6F_2}{\sqrt{61}}	+\dfrac{12F_3}{\sqrt{41}} =0 \\
-\dfrac{30F_2}{\sqrt{61}}	-\dfrac{12F_3}{\sqrt{41}} + 6F=0 \\
\end{array}
\right.
$; donc $-\dfrac{24F_2}{\sqrt{61}} + 6F=0 $ et 
${F_2} = F \dfrac{\sqrt{61}}{4}$.

Enfin, 
$F_3 = -\dfrac{F_2\sqrt{41}}{2\sqrt{61}}$ 
$= -F \dfrac{\sqrt{61}}{4} \dfrac{\sqrt{41}}{2\sqrt{61}}$
$= -F  \dfrac{\sqrt{41}}{8}$
\end{corrige}
\else
\fi


\question{Réaliser l’application numérique.}
\ifprof
\begin{corrige}
$F_2 = \SI{58576}{N}$, $F_3 = F_4 = -\SI{24011}{N}$, 
\end{corrige}
\else
\fi

\question{Les exigences du cahier des charges sont-elles vérifiées ?}
\ifprof
\begin{corrige}
Dans les barres 3 et 4 l'action maximale n'est pas dépassée. 
De plus, le vérin hydraulique est adapté à l'effort.
Les deux exigences sont donc vérifiées. 

\end{corrige}
\else
\fi

\ifcolle
\else
Eléments de correction :
\begin{enumerate}
\item $h=0$.
\item $F_2 = F \dfrac{\sqrt{61}}{4}$, $F_3 = F_4 = -F  \dfrac{\sqrt{41}}{8}$.
\item .
\item .
\end{enumerate}
\fi